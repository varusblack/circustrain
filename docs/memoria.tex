%% LyX 1.6.7 created this file.  For more info, see http://www.lyx.org/.
%% Do not edit unless you really know what you are doing.
\documentclass[spanish]{report}
\usepackage[T1]{fontenc}
\usepackage[latin9]{inputenc}
\setcounter{secnumdepth}{3}
\setcounter{tocdepth}{3}

\makeatletter

%%%%%%%%%%%%%%%%%%%%%%%%%%%%%% LyX specific LaTeX commands.
%% Because html converters don't know tabularnewline
\providecommand{\tabularnewline}{\\}

\makeatother

\usepackage{babel}
\addto\shorthandsspanish{\spanishdeactivate{~<>}}

\begin{document}

\title{El tren del circo}


\author{Bay�n Benegas, Marc \and Cuenca G�mez, Emilio \and Daza Pastrana,
Francisco Jos� \and Espinaco Villalba, Francisco Javier \and Tristancho
Reyes, �lvaro \and Vi�as Sandiez, Antonio Jes�s}

\maketitle

\chapter*{\index{URL del video}URL del video}

http://www.youtube.com/watch?v=Nyde3nEXl7s


\chapter*{Planificaci�n temporal}
\begin{itemize}
\item D�a 4 de octubre de 2010 a las 8:35, coordinaci�n
\item D�a 5 de octubre de 2010 a las 8:35, coordinaci�n
\item D�a 7 de octubre de 2010 a las 9:30, entrega iteraci�n 1
\item D�a 7 de octubre de 2010 a las 11:45, coordinaci�n
\item D�a 13 de octubre de 2010 a las 8:30, coordinaci�n
\item D�a 18 de octubre de 2010 a las 9:00, entrega iteraci�n 2
\item D�a 11 de noviembre de 2010 a las 21:00, entrega iteraci�n 3
\item D�a 9 de diciembre de 2010 a las 23:00, entrega iteraci�n 4
\end{itemize}

\chapter*{Seguimiento}

\begin{tabular}{|c|c|c|c|c|c|c|c|c|}
\hline 
\textit{Miembros\textbackslash{}Puntos por Iteraci�n} & \textit{1} & \textit{2} & \textit{3} & \textit{4} & \textit{5} & \textit{6} & \textit{7} & \textbf{\textit{TOTAL}}\tabularnewline
\hline
\hline 
\textit{Bay�n Benegas, Marc} & 5  & 5 & 5 & 5 &  &  &  & 20\tabularnewline
\hline 
\textit{Cuenca G�mez, Emilio} & 5  & 5 & 5 & 5 &  &  &  & 20\tabularnewline
\hline 
\textit{Daza Pastrana, Francisco Jos�} & 5 & 5 & 5 & 5 &  &  &  & 20\tabularnewline
\hline 
\textit{Espinaco Billalba, Francisco Javier} & 5 & 5 & 5 & 5 &  &  &  & 20\tabularnewline
\hline 
\textit{Tristancho Reyes, �lvaro} & 5 & 5 & 5 & 5 &  &  &  & 20\tabularnewline
\hline 
\textit{Vi�as Sandiez, Antonio Jes�s} & 5 & 5 & 5 & 5 &  &  &  & 20\tabularnewline
\hline 
\textbf{\textit{TOTAL}} & 30 & 30 & 30 & 30 &  &  &  & 120\tabularnewline
\hline
\end{tabular}


\chapter*{Iteraciones}


\section*{A�adidos en la iteraci�n 4:}
\begin{itemize}
\item Implementaci�n de cartas de acci�n.
\item Implementaci�n de las distintas bolsas (PerformanceBag, TalentBag).
\item Implementaci�n de las clases relativas al tablero (Board, City).
\item Implementaci�n de los diferentes comandos necesarios.
\item Implementaci�n de la clase que contiene la secuencia de juego (CircusTrainGame).
\item Implementaci�n de los distintos tipos de actuaciones (BankruptCircus,


PerformanceDemand,VictoryPoints).

\item Implementaci�n de la clase correspondiente al jugador (PlayerImpl).
\item Implementaci�n de los distintos talentos.
\item Documentos de asignaci�n de responsabilidades de los m�todos.
\item Memorandos t�cnicos.
\end{itemize}

\section*{Asuntos pendientes:}
\begin{itemize}
\item Terminar pruebas unitarias.
\item Refactorizar.
\item Implementaci�n de los modos de juego b�sico.
\item Documentos de asignaci�n de responsabilidades.
\item Integraci�n de las diferentes partes.
\end{itemize}

\chapter*{URL de implementaci�n}

http://code.google.com/p/circustrain/


\chapter*{Pruebas unitarias}
\begin{itemize}
\item Pruebas para Board
\item Pruebas para el comando CommandPerformance
\end{itemize}

\chapter*{Extensiones}

\textbf{\huge Modelo de an�lisis}
\end{document}
